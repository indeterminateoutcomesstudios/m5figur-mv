%
% m5bsp-mv.tex
%
% Diese Datei ist ein Beispielfigurenblatt des
% Midgard5-TeX-Figurenblatt-Projekts von Martin Väth.
% Sie muss zusammen mit m5figur-mv.cls und dem README.md weitergegeben werden;
% in m5figur-mv.cls steht auch die Lizenz.
%
% Die Datei m5bsp-mv.tex selbst darf nur vom Autor verändert werden.
% (Sie darf nur verändert werden, wenn sie umbenannt wird; das selbe
% gilt für m5figur-mv.cls)
%
% Um eigene Figuren anzufertigen, empfiehlt es sich daher,
% diese Datei zu kopieren und geeignet anzupassen.
% Da diese Datei nur ein Beispiel ist, empfiehlt es sich, nur das
%  "kompakte" Beispiel zwischen --- Anfang --- und --- Ende --- zu
% kopieren (oder beim Kopieren den Rest gleich wieder zu löschen),
% und bei Bedarf in diesem File hier nachzuschauen bzw. gezielt
% spezielle Zeilen zusätzlich zu kopieren.
%
% Ein pdflatex auf diese Datei erzeugt daraufhin die Figur(en) im PDF-Format;
% pdflatex muss dazu auf m5bsp-mv.cls zugreifen können. Normalerweise ist
% dies der Fall, wenn sich m5figur-mv.cls im selben Verzeichnis wie die
% Datei selbst befindet, aber es kann auch in einem geeigneten systemweiten
% Verzeichnis sein, wenn dieses dem TeX-System entsprechend bekannt gemacht
% wird.
%
% Vorteile der Ausarbeitung eines Figurenblatts mit diesem Projekt gegenüber
% der reinen PDF- oder Papierversion sind:
%
% 1. Gemeinsame Daten müssen nur einmalig für alle 3 Blätter eingegeben werden.
% 2. Es ist kein Problem, z.B. mehrfache Zauberblätter auszudrucken (falls
%    ein Blatt nicht alle Informationen aufnehmen kann).
% 3. Tabellen werden den Daten angepasst (etwa falls die Beschreibung einer
%    Wirkung bei Zaubern lange ist, oder eine Fertigkeit Zusatzbemerkungen
%    erfordert).
% 4. Man kann mehrere Figuren mit einer einzigen Datei verwalten und muss
%    dabei für "ähnliche" Figuren (z.B. die selbe Figur in verschiedenen
%    Graden) die identischen Daten (etwa Basiswerte) nicht mehrfach eingeben.
%    Falls sich Basis- oder Fertigkeitswerte ändern (z.B. durch
%    Verbesserungswürfe oder Lernen), kann man nur den geänderten Wert
%    neu setzen, ohne alle anderen zu wiederholen.
%    Selbstverständlich kann man auch einfach neue Fertigkeiten hinzufügen.
% 5. Automatische Berechnung von Werten!
%
% In der Tat, für die Fertigkeits- und Waffenlisten auf dem Abenteuerblatt
% der Figuren müssen die Einträge aus dem Figurenblatt nicht ganz wiederholt
% werden, sondern sie können referenziert werden, und die Berechnung des
% Erfolgs-/Trefferwerts geschieht normalerweise automatisch aufgrund der
% anderen Daten.
% Dabei wurde das Hauptaugenmerk auf die Flexibilität gerichtet.
% Beispielsweise können Modifikationen (etwa für magische Waffen)
% berücksichtigt werden, und ggf. kann man das Ergebnis auch fest angeben
% (oder z.B. leer lassen).



\documentclass{m5figur-mv}[2016/01/04]
% Der Datumsteil [...] ist optional, aber wir stellen damit "sicher", dass
% eine aktuelle Version von m5figur-mv.cls benutzt wird (falls auf dem Rechner
% eine veraltete vorhanden sein sollte, wird eine Warnung gedruckt).
%
% Bei anderer als UTF8-Kodierung für Umlaute und Sonderzeichen
% (es gibt z.B. ansinew (Windows), latin1 (Linux), applemac (Mac); dies
% bedeutet, dass Sie die Umlaute (äöüÄÖÜ), scharfes ß, Paragraph (§)
% und Eurozeichen (€) in dieser Datei nicht richtig gelesen und
% mit Ihrem Texteditor oder anderen Programm konvertiert haben),
% muss ggf. die Kodierung als Option angegeben werden, also z.B. so:
% \documentclass[ansinew]{midgard5}

% Falls Sie keine geeignete Kodierung finden, können Sie die folgenden
% deutschen TeX-Kürzel benutzen, die immer funktionieren:
% "a "o "u  "A "O "U "s (Paragraph und Euro werden wohl nicht benötigt).

% Gänsefüßchen schreibt man in TeX immer so, unabhängig von der Kodierung:
% "`Text in G"ansef"u"schen"'

% Andere spezielle TeX-Eigenarten (spezielle Trennregeln u.ä.) entnehmen
% Sie bitte einer TeX-Anleitung.

\begin{document}

% Das Setzen der Daten sollte erst nach \begin{document} erfolgen,
% da vorher keine Umlaute und deutschen Trennregeln benutzt werden können.

% Die wichtigsten Daten werden mit \DefData gesetzt.
% Die Daten sind normalerweise einfache Werte, manchmal jedoch Tupel,
% wie etwa bei Gewandtheit, in der der normale Wert und der in Rüstung
% angegeben werden muss. Selbst wenn die Werte identisch sein sollten,
% müssen in diesen Fällen beide Werte angegeben werden.
% Das folgende Beispiel ist insofern vollständig, als alle Daten vorkommen,
% und daher für jeden Wert klar ist, ob er als Tupel oder alleinstehend
% gesetzt werden muss.
% Die *Reihenfolge*, in der die Daten angegeben werden, spielt keine Rolle.


\DefData{Spieler}{Martin Väth}

% Ungelernte Fertigkeiten mit oder ohne Leiteigenschaft?
% \UngelerntMit ist die Voreinstellung
\UngelerntMit
%\UngelerntOhne

{ % Alle Daten innerhalb von Klammern werden nach deren Schließen vergessen.
% Dies kann man benutzen, um z.B. mehrere Figuren zu erstellen.
% Hätten wir den Spielernamen ebenfalls in die Klammer gesetzt, würde dieser
% Name anschließend ebenfalls vergessen. So bleibt er "global"
% (bis er "manuell" anders gesetzt wird, siehe weiter unten).

% \ClearLists löscht alle Listen (Fertigkeiten, Waffen, Zauber, Gegenstände).
% Dies ist hier eigentlich redundant, da die Listen zu Beginn leer sind:
\ClearLists

% Zunächst die Längen der Blätterköpfe:
% Die folgenden Werte sind die Voreinstellung
\setlength{\FigurenblattKopf}{50mm}
\setlength{\AbenteuerblattKopf}{40mm}
\setlength{\ZauberblattKopf}{40mm}

% Der Trenner/Zwischenraum zwischen verschiedenen Zaubern:
% Der folgende Wert ist die Voreinstellung
\renewcommand*{\ZauberTrenner}{\vspace*{2mm}}

\DefData{Figur}{Cynyr ap Clai}
\DefData{Typ}{Schamane}
\DefData{Grad}{17}
\DefData{Spezialisierung}{}
\DefData{St}{73}
\DefData{Gs}{96}
\DefData{Gw}{{85}{85}}% Normal bzw. in Rüstung
\DefData{Ko}{87}
\DefData{In}{92}
\DefData{Zt}{83}
\DefData{Au}{94}
\DefData{pA}{90}
\DefData{Wk}{86}
\DefData{B}{{29}{29}}% Normal bzw. in Rüstung
\DefData{GiT}{\StandardGiT}% Das ist Voreinstellung
\DefData{Raufen}{9}
\DefData{Ausdauerbonus}{11}
\DefData{Schadensbonus}{3}
\DefData{Angriffsbonus}{{2}{2}}% Normal bzw. in Rüstung
\DefData{Abwehrbonus}{{1}{1}}% Normal bzw. in Rüstung
\DefData{ResistenzBonus}{{3}{3}}
\DefData{Zauberbonus}{1}
\DefData{RK}{{LR}{2}}
\DefData{Ruestung}{Lederhalsschutz Lederstiefel}
\DefData{Sehen}{{0}{6}}
\DefData{Nachtsicht}{{0}{0}}
\DefData{Hoeren}{{0}{6}}
\DefData{Riechen}{{4}{10}}
\DefData{SechsterSinn}{{2}{8}}
\DefData{LP}{19}
\DefData{AP}{55}
\DefData{GG}{2}
\DefData{SG}{10}
\DefData{Geburtsdatum}{}
\DefData{Alter}{28}
\DefData{haendig}{rechts}
\DefData{Groesse}{188}
\DefData{Gestalt}{g/n}
\DefData{Gewicht}{75}
\DefData{Stand}{Mittelschicht}
\DefData{Heimat}{Fuardayn}
\DefData{Glaube}{schamanisch}
\DefData{Merkmale}{Totem: Wolf}
\DefData{Datum01}{{23.12.14}{5450}{590}{1348}}
\DefData{Datum02}{{30.12.14}{5750}{20}{448}}
\DefData{Datum03}{{29.12.16}{6000}{180}{2263}}
\DefData{Datum04}{{15.04.17}{6200}{380}{3063}}

% Leere Definitionen sind eigentlich redundant, aber zur Vollständigkeit:

\DefData{Datum05}{{}{}{}{}}
\DefData{Datum06}{{}{}{}{}}
\DefData{Datum07}{{}{}{}{}}
\DefData{Datum08}{{}{}{}{}}
\DefData{Datum09}{{}{}{}{}}
\DefData{Datum10}{{}{}{}{}}
\DefData{Datum11}{{}{}{}{}}
\DefData{Datum12}{{}{}{}{}}
\DefData{Datum13}{{}{}{}{}}
\DefData{Datum14}{{}{}{}{}}
\DefData{Datum15}{{}{}{}{}}
\DefData{Datum16}{{}{}{}{}}
\DefData{Datum17}{{}{}{}{}}
\DefData{Datum18}{{}{}{}{}}
\DefData{Datum19}{{}{}{}{}}
\DefData{Datum20}{{}{}{}{}}

% Praxispunkte für Zauber:

\DefData{Beherrschen}{}
\DefData{Erkennen}{}
\DefData{Formen}{}
\DefData{Zerstoeren}{}
\DefData{Dweomer}{}
\DefData{Bewegen}{}
\DefData{Erschaffen}{}
\DefData{Veraendern}{}
\DefData{Wundertaten}{1}
\DefData{Zauberlieder}{---} % Der (lange) Gedankenstrich


% Hier setzen wir die Daten für die Basisfertigkeiten und Boni gemischt:
% Es ist übersichtlicher, dies zusammen zu halten.
% Der Aufruf von \Fertigkeit wird gleich erklärt.
% Beachte, dass durch den Aufruf von \Fertigkeit bereits an dieser Stelle
% schon die ersten Einträge in der ersten Fertigkeitstabelle des
% Figurenblattes festgelegt werden.

\Fertigkeit1{Abwehr}[15]
\Fertigkeit1{Zaubern}[17]
\Fertigkeit1{Resistenz}[15]

% Nun kommen die Basisfertigkeiten mit Boni.
% Die folgenden Werte sind die Voreinstellung, und wir könnten diese Befehle
% daher weglassen: % Sie geben die Werte entsprechend der oben gesetzten
% Fertigkeiten und Boni aus.

\DefData{Abwehr}{{\StandardAbwehrPlus{0}}% Normal
	{\StandardAbwehrPlus{0}}}% Mit Verteidigungswaffe
\DefData{Resistenz}{{\StandardResistenz1}% Gegen Körpermagie
	{\StandardResistenz2}}% Gegen Geistesmagie
\DefData{Zaubern}{\StandardZaubern}


% Mit dem folgenden Kommando machen wir einen Eintrag in der Fertigkeitsliste
% auf dem Figurenblatt und schaffen zugleich eine Referenz für spätere Blätter.
% Die Reihenfolge der Aufrufe ist im Prinzip beliebig, sie kann sogar mit
% später erklärten Aufrufen für andere Listen gemischt werden.
% Die Reihenfolge spielt allerdings insofern eine Rolle, dass frühere Aufrufe
% zuerst in der Liste erscheinen.

%\Fertigkeit1[Name]{Fähigkeit}[Bonus][PP]
% 1/2 steht für die erste/zweite Liste auf dem *Figurenblatt*.
% [Name] ist der Name, unter dem die Fähigkeit später referenziert werden kann;
% er muss eindeutig sein und darf keine Umlaute/Sonderzeichen enthalten.
% Falls Name leer ist, kann nicht referenziert werden;
% dies ist z.B. bei leeren Einträgen sinnvoll.
% Falls [Name] weggelassen wird, wird Fähigkeit als Name benutzt.
% (Dies kann bei komplizierten Fähigkeiten - mit Umlauten/Sonderzeicehn usw. -
% zu TeX-Fehlern führen; daher [Name] im Zweifelsfall nicht weglassen, sondern
% ein einfaches (aber eindeutiges) Wort dafür benutzen).
%
% [Bonus] ist der Bonus, auf den die Fertigkeit gelernt ist,
% [PP] die ev. noch ausstehende Praxispunkte.
% [Bonus] und [PP] sind optional; wird jedoch [Bonus] weggelassen, muss auch
% [PP] weggelassen werden.

\Fertigkeit1[]{}% freier Eintrag; [] bedeutet, dass nicht referenziert wird
\Fertigkeit1{Wahrnehmung}[6]
\Fertigkeit1{Pflanzenkunde}[11]
\Fertigkeit1[Fuardayn]{Landeskunde Fuardayn}[11]% Referenzname: Fuardayn
\Fertigkeit1[Orkland]{Landeskunde Orkland}[10][1]% Referenzname: Orkland
\Fertigkeit1{Tierkunde}[9][2]
\Fertigkeit1{Zauberkunde}[11]
\Fertigkeit1[Gebirge]{Überleben Gebirge}[9]
\Fertigkeit1[Wald]{Überleben Wald}[8]
\Fertigkeit1[Geraete]{Gerätekunde}[8]
\Fertigkeit1[]{}
\Fertigkeit1{Erste Hilfe}[15][8]
\Fertigkeit1{Trinken}[8]
\Fertigkeit1{Verstellen}[8]
\Fertigkeit1[]{}
\Fertigkeit1[Anfuehren]{Anführen}[8]% Referenzname: Anfuehren
\Fertigkeit1{Seilkunst}[13]
\Fertigkeit1{Schleichen}[9]
\Fertigkeit1{Spurensuche}[8]
\Fertigkeit1{Reiten}[12]
\Fertigkeit1{Bootfahren}[13]
\Fertigkeit1{Schwimmen}[16][1]
\Fertigkeit1{Klettern}[14][1]
\Fertigkeit1[]{}
\Fertigkeit1[]{}
\Fertigkeit1[Wurf daneben]{Wurf daneben\tiny\newline
% Wir benuzen "minipage" um eine "overfull box" zu erzwingen, die
% auch Einträge rechts in der Tabelle überschreiben kann
\begin{minipage}{41mm}\raggedright PW-25:Zt/2; W.-Dauer Grad/5 x 10\,sec;
\newline
Bei Fehler: 1W6 LP+AP\end{minipage}}


% Für Waffenfertigkeiten lohnen sich Abkürzungen in den Referenznamen,
% da diese in der Praxis häufig referenziert werden müssen:

\Fertigkeit2[Stich]{Stichwaffe}[5]
\Fertigkeit2[1Schlag]{Einhandschlagwaffe}[8]
\Fertigkeit2[1Schwert]{Einhandschwert}[8][4]
\Fertigkeit2[Stielwurf]{Stielwurfwaffe}[6]
\Fertigkeit2[Fecht]{Fechtwaffe}[5]
\Fertigkeit2[Bogen]{Bögen}[4]
\Fertigkeit2[]{}
\Fertigkeit2[]{}
\Fertigkeit2[]{}
\Fertigkeit2[]{}
\Fertigkeit2[]{}
\Fertigkeit2[]{}
\Fertigkeit2[]{}
\Fertigkeit2[]{}
\Fertigkeit2[]{}
\Fertigkeit2[]{}
\Fertigkeit2[]{}
\Fertigkeit2{Orkisch}[10]
\Fertigkeit2{Chryseisch}[11]
\Fertigkeit2{Rawindi}[11]
\Fertigkeit2{Waelska}[13]
\Fertigkeit2{Albisch}[13]
\Fertigkeit2{Twyneddisch}[18]
\Fertigkeit2{Zauberschrift}[18]


% Referenzierbare Fertigkeiten können nach dem Aufruf von \Fertigkeit...
% noch modifiziert werden. Dies ist beispielsweise dann sinvoll, wenn wir
% ein zweites Figurenblatt der selben Figur in höherem Grad drucken:
% Wir wollen dann i.d.R. nicht alle \Fertigkeit...-Kommandos erneut eingeben,
% sondern nur z.B. Boni ändern (und ggf. *zusätzliche* \Fertigkeit-Kommandos)
% einfügen.
%
% Die Macros zum Modifizieren der Fertigkeiten lauten:
%
% \RenewBeschreibung{Name}{Fertigkeit}
% \RenewWert{Name}{neuer Wert}
% \RenewPP{Name}{neuer PP}
%
% Es ist sogar möglich, Referenzen zu erstellen für Einträge, die gar nicht
% mit \Fertigkeit gelistet wurden (für Raufen geschieht dies z.B. automatisch).
% Dies geht mit einem der beiden folgenden Macros:
%
% \DefReferenz{Name}{Fertigkeit}{Wert}{PP}
% \DefReferenzWertFirst{Wert}{Name}{Fertigkeit}{PP}
%
% Ein Beispiel wird weiter unten gegeben.

% Das folgende Kommando ist ähnlich wie \Fertigkeit, betrifft allerdings
% die Fertigkeitsliste des *Abenteuerblattes*. Zudem werden Einträge aus
% \Fertigkeit referenziert (und keine neuen Referenzen erstellt).
% Für die korrekte Referenzierung ist es nicht nötig, das Figurenblatt
% später auch tatsächlich auszugeben. Zudem ist es nicht nötig, dass der
% entsprechende \Fertigkeit-Aufruf vor der Referenzierung im File steht.
% Er muss allerdings vor der Ausgabe des Abenteuerblatts erscheinen.

%\Fertig1[Beschreibung]{Referenz}[Modifikation]{Leiteigenschaft/Wert}[PP]
% 1/2 steht für die erste/zweite Liste auf dem *Abenteuerblatt*.
% [Beschreibung] ist optional; wenn sie fehlt, wird die der Referenz genommen.
% {Referenz} ist nicht optional; sie kann aber leer bleiben, also {};
% in dem Fall sollten [Beschreibung], Wert und [PP] angegeben werden;
% falls {Referenz} nicht leer ist, muss ein entsprechender Eintrag mit
% \Fertigkeit erfolgt sein (Ausnahme: Raufen, siehe unten).
% [Modifikation] ist eine optionale Modifikation des errechneten Werts.
% Der Wert wird aus Referenz und Leiteigenschaft berechnet, wenn er nicht
% explizit angegeben ist. Eine leere Leiteigenschaft bedeutet, dass keine
% existiert; um eine Leiteigenschaft oder einen leeren Wert als *Wert* zu
% interpretieren, kann man nochmals klammern. Z.B. ist {{}} ein leerer Wert.
% Die spezielle "Leiteigenschaft" AB bedeutet, den Angriffsbonus zu nehmen.
% [PP] ist optional; wenn der Eintrag fehlt wird der Wert der Referenz
% genommen.
% Die "Fähigkeit" Raufen kann automatisch referenziert werden.

\Fertig1{Wahrnehmung}{}
\Fertig1{Pflanzenkunde}{In}
\Fertig1{Fuardayn}{In}
\Fertig1{Orkland}{In}
\Fertig1{Tierkunde}{In}
\Fertig1{Zauberkunde}{In}
\Fertig1{Gebirge}{In}
\Fertig1{Wald}{In}
\Fertig1{Geraete}{In}
\Fertig1{}{}
\Fertig1{Erste Hilfe}{Gs}
\Fertig1{Trinken}{}
\Fertig1{Verstellen}{pA}
\Fertig1{}{}
\Fertig1{Anfuehren}{pA}
\Fertig1{Seilkunst}{Gs}
\Fertig1{Schleichen}{Gw}
\Fertig1{Spurensuche}{In}
\Fertig1{Reiten}{Gw}
\Fertig1{Bootfahren}{Gs}
\Fertig1{Schwimmen}{Gw}
\Fertig1{Klettern}{St}
\Fertig2{Wurf daneben}{}
\Fertig2{}{}
\Fertig2{}{}
\Fertig2{Orkisch}{In}
\Fertig2{Chryseisch}{In}
\Fertig2{Rawindi}{In}
\Fertig2{Waelska}{In}
\Fertig2{Albisch}{In}
\Fertig2{Twyneddisch}{In}
\Fertig2{Zauberschrift}{Zt}


% Das folgende Kommando ist ganz ähnlich zu \Fertig, aber für Waffen:

%\Waffe1[Beschreibung]{Referenz}[Modifikation]{AB/leer}[Schaden][Nah]
% Wie bei \Fertig, nur dass AB für den Angriffsbonus steht, und dass [Schaden]
% und [Nah] der Schaden bzw Nahkampfschaden ist; beides ist optional.
% Wird allerdings [Schaden] ausgelassen, muss auch [Nah] ausgelassen werden.
% Die "Waffe" Raufen kann automatisch referenziert werden.
%
% Für den Schaden kann das Macro \PlusSB{Wert} benutzt werden:
% Dies führt dazu, dass die Summe "Wert + Schadensbonus" mit einem
% davorgestellten "+" (bzw. "-" im negativen Fall) ausgegeben wird;
% falls die Summe 0 ist, wird nichts ausgegeben.
% Dadurch sind Angaben wie beispielsweise "1W6\plusSB{-1}" bei einem Dolch,
% "1W5\plusSB{}" bei einem Kurzschwert (der leere Eintrag hat den Wert 0)
% oder "1W5\plusSB{1}" bei einem Langschwert sinnvoll.
% Beachten Sie, dass die Benutzung dieses Macros nicht immer sinnvoll ist:
% Bei einer Fernwaffe beispielsweise wird der Schadensbonus ja nicht addiert.

\Waffe1[Wasserklinge]{1Schwert}[2]{AB}[1W6\PlusSB{2}]
\Waffe1[Langschwert]{1Schwert}{AB}[1W6\PlusSB{1}]
\Waffe1[Keule]{1Schlag}{AB}[1W6\PlusSB{-1}]
\Waffe1[Dolch]{Stich}{AB}[1W6\PlusSB{-1}]
\Waffe1[Wurfkeule]{Stielwurf}{AB}[1W6]
\Waffe1[Bogen]{Bogen}{AB}[1W6]
\Waffe1{}{}
\Waffe1{}{}
\Waffe2{}{}
\Waffe2{}{}
\Waffe2{}{}
\Waffe2{}{}
\Waffe2{}{}
\Waffe2{}{}
\Waffe2{}{}
\Waffe2{Raufen}{}

{ % Hier klammern wir nochmals, um die folgenden Daten lokal zu machen


% Für das Zauberblatt brauchen wir noch ein Kommando, um die Zauber zu
% setzen. Dies ist das folgende:

%\Zauber1{Name}{AP}{Prozess}{Zauberdauer}{Reichweite}{Wirkungsziel}%
%       {Wirkungsbereich}{Wirkungsdauer}{Beschreibung}{Art}

\Zauber1{Hitzeschutz Kälteschutz}{1}{Verändern}{Augenblick}{-}{Körper}%
  {Zauberer}{2\,min}{S.~94}{Gedanken}
\Zauber1{Segnen}{2}{Wund./""Veränd.}{1\,min}{Berührung}{Körper}%
  {1 Wesen}{10\,min}{S.~146; +1 Erfolgs/Widerstands "~5 Prüfwürfe}{Geste}
\Zauber1{Schwäche}{1 pro Wesen}{Verändern}{Augenblick}{30\,m}{Körper}%
  {1--10 Wesen}{2\,min}{S.~110; Stärke "~20, Raufen "~1, Schaden "~1}%
  {Eschenrinde (1GS)}
\Zauber1{Schwingenkeule}{1}{Wund./""Formen}{10\,sec}{Berührung}%
  {Umgebung}{1 Objekt}{2\,min}{S.~146}{Geste}
\Zauber1{Felsenfaust}{1}{Verändern}{10\,sec}{Berührung}{Körper}%
  {1 Wesen}{2\,min}{S.~81; +Hand/""Faustkampf (1W6+""SB);
  nicht: Geste, HG}{Kiesel (1SS)}
\Zauber1{Verfluchen}{2}{Wund./""Veränd.}{10\,sec}{15\,m}{Körper}%
  {1 Wesen}{10\,min}{S.~147; "~1 Erfolgs/Widerstands +5 Prüfwürfe}{Geste}
\Zauber1{Macht über das Selbst}{1}{Beherrschen}{10\,sec}{-}{Körper}%
  {Zauberer}{24\,h; Konz.}{S.~98}{Gedanke}
\Zauber1{Bannen von Licht}{1}{Zerstören}{1\,sec}{0\,m}{Umgebung}%
  {9\,m Umkreis}{bis 10\,min}{S.~66; bewegt sich mit Zauberer;
  nicht gegen Infrarotsicht}{Wort}
\Zauber1{Unsichtbarkeit}{4 (oder 8)}{Verändern}{10\,sec}{-}{Körper}%
  {Zauberer}{10\,min; Konz.}{S.~120}{Wort}
\Zauber2{Bärenwut}{2}{Dweomer/""Verändern}{Augenblick}{-}{Körper}%
  {Zauberer}{2\,min}{S.~149; nur bei LP-Verlust anwendbar;
  Stärke +30, Raufen +4, ohne AP wie normal bis weniger als 6~LP}{Geste}
\Zauber2{Heilen von Wunden}{3}{Wund./""Ersch.}{1\,min}{Berührung}%
  {Körper}{1 Wesen}{0}{S.~144; 1W6 LP+AP (1 mal pro 3 Tage)}{Geste}
\Zauber2{Naturgeist rufen}{3}{Dweomer/""Beherrschen}{5\,min}{500\,m}{Geist}%
  {-}{2\,min}{S.~256}{Trommel}
\Zauber2{Schlaf}{4 je Wesen}{Beherrschen}{10\,sec}{30\,m}{Körper}%
  {bis 6 Wesen}{8\,h}{S.~109; bis max.\ Grad 10}{Lotusblütenstaub\,10GS}


% Ebenfalls auf dem Zauberblatt stehen die Liste der magischen Gegenstände.
% Diese wird mit folgendem Kommando gefüllt:

%\Magisch{Gegenstand}{Beschreibung}

\Magisch{1 Kraut der konz.\ Energie}{volle AP;
  später verliert man 3 LP}
\Magisch{1 Krafttrunk}{2W6 AP}
\Magisch{2 Heiltrank}{1W6 LP+AP}
\Magisch{5 Berserkerpilze}{}
\Magisch{2 Amphoren Zauberöl}{}
\Magisch{30 Portionen Rauschkraut}{von reisendem Zwergenhändler
  aus Wagen gekauft}
\Magisch{1 Phiole Weihwasser}{}


% Es folgt die eigentliche Ausgabe der gewünschten Blätter:

{ % Diese Klammer dient dazu, dass das Folgende nur lokal wirkt:
% Für das Figurenblatt entfernen wir die ausführliche Beschreibung für
% die Fertigkeit "Wurf daneben"
\RenewBeschreibung{Wurf daneben}{Wurf daneben}

\Output{Figurenblatt}
}% Nach dieser Klammer ist die Änderung von "Wurf daneben" wieder vergessen...

\Output{Abenteuerblatt}% Ausgabe des Abenteuerblattes
% Zu Demonstrationszwecken wird auf diesem Blatt der Text "Veraltete Version"
% an einer festen Koordinate (Von der linken oberen Ecke aus gesehen
% 17cm nach unten und 2cm rechts eingefügt).
% Ähnlich können natürlich ggf. ganze Textpassagen oder Tabellen an
% gewünschten Stellen eingefügt werden (z.B. in
% \begin{minipage}{10cm} .... \end{minipage}
% eingeschlossen).
\vbox to0pt{\vspace*{17cm}\hbox to0pt{\hspace*{2cm}Veraltete Version}}
\Output{ZauberblattMagier}% Ausgabe des Zauberblattes mit Bemerkung für Magier
%\Output{Zauberblatt}% Ausgabe des Zauberblattes ohne Bemerkung für Magier

% Wir fügen zum Zauberblatt noch etwas hinzu (weil die Figur inzwischen mehr
% trainiert hat) und drucken die neue Version nochmals:

\Zauber1{Heranholen}{1}{Bewegen}{10\,sec}{30\,m}{Umgebung}%
  {1 Objekt}{10\,sec}{S.~91; WW:Abwehr + Stärke/5}{Geste}
\Zauber1{Dinge wiederfinden}{1}{Erkennen}{10\,min}{unbegrenzt}{Geist}%
  {1 Objekt}{10\,min}{S.~73; 20 Grad genau, Zehnerpotenz von m}{Gedanke}
\Zauber1{Rindenhaut}{2}{Verändern}{10\,sec}{Berührung}{Körper}%
  {1 Wesen}{10\,min}{S.~157; B-3 wie KR (-3 LP/AP)}{Eich.Rinde 2SS}

\Zauber2{Lebensstärkung}{4}{Dweomer/""Erschaffen}{20\,sec}{Berührung}{Körper}%
  {1 Wesen}{30\,min}{S.~154; 1W6+Grad AP, W.-Dauer auch über Max,
  1 mal pro Tag}{Geste}
\Zauber2{Schlachtenwahnsinn}{6}{Dweomer/""Ver\"{a}ndern}{10\,sec}{-}{Körper}%
  {Zauberer}{1\,min}{S.~157; Bärenwut+Beschleunigen (S.~149,~68),
  Gegner -2 auf EW:Angriff}{Geste}

%\Output{ZauberblattMagier}% Ausgabe des Zauberblattes mit Bemerkung für Magier
\Output{Zauberblatt}% Ausgabe des Zauberblattes ohne Bemerkung für Magier

% Wollte man statt fester Koordinaten den Text einfach am Ende ausgeben,
% muss man verhindern, dass "\Output{Zauberblatt}" die Seite beendet.
% Dies könnte man so tun:

% {\def\clearpage{\newline}Output{Zauberblatt}}
% \vbox to0pt{\vspace*{17cm}\hbox to0pt{\hspace*{2cm}Version 2}}


} % Schließen der vorherigen Klammer zum "Vergessen" der lokalen Daten;
% in diesem Fall waren dies alle Einträge mit \Zauber und \Magisch.
% Wir benutzen dies, um noch ein anderes Zauberblatt (mit anderen Einträgen)
% für die selbe Figur zu erstellen - die anderen Daten sind ja noch die selben.

\Zauber1{Wandelhand}{2}{Bewegen}{5\,sec}{Berührung}{Umgebung}%
  {1 Objekt}{10\,min}{M4:S.~184; 6LP $\infty$AP Gw80 St50 B12 11/10/12
  Raufen +6(W-3) Würgen (1W6-1 AP; ab 6 Rd.\ PW+10:Ko)
  Angr.\ -4 oder beidh.\ losreißen}{Wort}
\Zauber1{Seelenheilung}{2}{Wund./""Veränd.}{10\,min}{3\,m}{Geist}%
  {1 Wesen}{0}{S.~146}{Wort}
\Zauber1{Wasseratmen}{2}{Verändern}{10\,sec}{Berührung}{Körper}%
  {1 Wesen}{8\,h}{S.~127}{Geste}
\Zauber1{Stille}{2}{Formen}{Augenblick}{0\,m}{Umgebung}%
  {3\,m Umkreis}{1\,min}{S.~115; bewegt sich mit Zauberer}{Eulenfedern (2GS)}
\Zauber1{Sumpfboden}{1}{Formen}{10\,sec}{50\,m}{Umgebung}%
  {15\,m Umkreis}{2\,min}{S.~117; B/4 (PW+20); Wagen bleiben stecken;
  keine Pferde; Reiter ggf.\ EW-4}{Wasserlinsen (1GS)}
\Zauber1{Seelenkompaß}{2}{Erkennen}{1\,min}{6\,km}{Geist}%
  {1 Wesen}{10\,min}{S.~112; benötigt Haar}{Silbernadel (10GS)}
\Zauber1{Tiergestalt (Wolf)}{6}{Dweomer/""Verändern}{20\,sec}{-}{Körper}%
  {Zauberer}{$\infty$}{S.~160}{8 Fellstücke des Wolfs (1GS)}
\Zauber1{Geisterlauf}{4 je Wesen}{Bewegen}{30\,min}{3\,m}{Körper}%
  {bis 7 Wesen}{variabel}{Ergänzungen S.~16, M4:S.~125}{8 Symbole (5GS)}
\Zauber1{Kraftspende}{1}{Dweomer/""Bewegen}{Augenblick}{30\,m}{Körper}%
  {1 Wesen}{0}{S.~153; 1W3+1 AP auch über Max. (1 mal pro Tag);
  nicht auf Zauberer}{Geste}
\Zauber2{Handauflegen}{1}{Wund./""Ersch.}{10\,sec}{Berührung}%
  {Körper}{1 Wesen}{0}{S.~143; 1W6 AP (1 mal pro Tag)}{Geste}
\Zauber2{Brot und Wasser}{3}{Erschaffen}{10\,min}{0\,m}{Umgebung}%
  {-}{0}{S.~71}{Mehlstaub \& Wassertropfen (1GS)}
\Zauber2{Hagel}{12}{Erschaffen}{30\,sec}{100\,m}{Umgebung}%
  {30\,m Umkreis}{2\,min; Konz.}{S.~89; +15(1W6); Schild +4 auf Abwehr}%
  {Diamanten (50GS)}
\Zauber2{Vision}{alle (mind.\ 3)}{Wund./""Erk.}{6\,h}{-}{Geist}%
  {Zauberer}{1\,h}{S.~147}{Kräuter}
\Zauber2{Schutzgeist}{3}{Dweomer/""Bewegen}{10\,sec}{Berührung}{Umgebung}%
  {Zauberer}{bis 30\,min}{S.~159;
  bei LP-Verlust letzten Wurf wiederholen}{Geste}
\Zauber2{Wasserstrahl (W.-Klinge)}{1 (+1)}{Erschaffen}%
  {Augenblick}{5\,m}{Umgebung}{Strahl}{1 AP für "`aus"'}{S.~128;
  +8(1W6+2), schwer: PW:Stärke (1W6-1\,m) 50\% für 1m Feuerwand}{Geste}
\Zauber2{Wasserlauf/-wandeln (W.-Klinge)}{3}{Bewegen}{20\,sec}{-}{Körper}%
  {Zauberer}{10\,min}{S.~127; B/2, sonst EW+8:Geländelauf}{Gedanke}
\Zauber2{Nebel schaffen (W.-Klinge)}{4}{Erschaffen}{20\,sec}{0\,m}%
  {Umgebung}{50\,m Umkreis}{30\,min}{S.~103; Sicht 10\,m tags/3\,m nachts,
  Geräusche nur halbweit}{Geste}

\Magisch{10 Duftwasser}{pA+10}
\Magisch{Springwurz}{}
\Magisch{Schutzamulett}{gegen Lauscher und Beobachter, ABW:15}
\Magisch{Delfinamulett}{Orden/Aura}
\Magisch{Wasserklinge}{}
\Magisch{Maske der Ferne}{500\,km, 30\,min, ABW:1}

% Anstelle des Standard-Befehls
%\Output{Zauberblatt} oder
%\Output{ZauberblattMagier}
% wird hier zu Demonstrationszwecken das Zauberblatt „manuell“ gebastelt,
% um noch eine Demo-Bemerkung einzufügen.

\Output{ZauberblattKopf}
\begin{minipage}{\Halbseitenbreite}%
\Output{ZauberA}% oder \Output{ZauberAMagier}
\scriptsize Die mit W.-Klinge markierten Zauber sind die der Wasserklinge
\end{minipage}%
\begin{minipage}{\Halbseitenbreite}%
\Output{ZauberB}% oder \Output{ZauberBMagier}
\Output{magischeGegenstaende}%
\end{minipage}%
\clearpage

}% Schließen der ersten Klammer zum "Vergessen" aller Figurendaten.

% Wird das Kommentarzeichen, vor dem folgenden \end{document} enfernt,
% hört das Dokument hier auf - ansonsten folgt die nächste Figur
%\end{document}

{% Eine zweite Figur.

% Wenn Sie selbst eine Figur mit m5figur-mv machen wollen, empfiehlt es sich,
% nur den folgenden Teil zwischen "--- Anfang ---" und "--- Ende ---" in eine
% neue Datei zu kopieren und die Kommentare/Daten entsprechend zu editieren.

% --- Anfang ---
% Laden von m5figur-mv, aktuelle Version fordern:
%\documentclass{m5figur-mv}[2015/01/06]
%\begin{document}

%\DefData{Spieler}{Martin Väth}

%\UngelerntMit
%\UngelerntOhne

% Da der Name etwas kürzer ist, machen wir die Box oben etwas kleiner:
\setlength{\FigurenblattKopf}{40mm}
\setlength{\AbenteuerblattKopf}{30mm}
\setlength{\ZauberblattKopf}{30mm}
\DefData{Figur}{Leo Anjini}
\DefData{Typ}{Glücksritter}
\DefData{Grad}{5}
\DefData{Spezialisierung}{Florett}
\DefData{St}{75}
\DefData{Gs}{98}
\DefData{Gw}{{59}{\begin{minipage}{10mm}59\\\tiny34 PR\end{minipage}}}
\DefData{Ko}{51}
\DefData{In}{91}
\DefData{Zt}{35}
\DefData{Au}{82}
\DefData{pA}{100}
\DefData{Wk}{78}
\DefData{B}{{23}{19}}% Normal bzw. in Rüstung
\DefData{GiT}{\StandardGiT}% Das ist Voreinstellung
\DefData{Raufen}{9}
\DefData{Ausdauerbonus}{8}
\DefData{Schadensbonus}{3}
\DefData{Angriffsbonus}{{2}{2}}% Normal bzw. in Rüstung
\DefData{Abwehrbonus}{{2}{0}}% Normal bzw. in Rüstung
\DefData{ResistenzBonus}{{1}{1}}
\DefData{Zauberbonus}{0}
\DefData{RK}{{\begin{minipage}{5mm}KR\newline PR\end{minipage}}%
	{\begin{minipage}{2mm}3\newline4\end{minipage}}}
\DefData{Ruestung}{Kettenhemd/mit Beinschienen}
\DefData{Sehen}{{2}{8}}
\DefData{Nachtsicht}{{0}{0}}
\DefData{Hoeren}{{0}{6}}
\DefData{Riechen}{{0}{6}}
\DefData{SechsterSinn}{{0}{6}}
\DefData{LP}{15}
\DefData{AP}{30}
\DefData{GG}{}
\DefData{SG}{2}
\DefData{Geburtsdatum}{}
\DefData{Alter}{}
\DefData{haendig}{rechts}
\DefData{Groesse}{177}
\DefData{Gestalt}{n/n}
\DefData{Gewicht}{75}
\DefData{Stand}{Mittelschicht}
\DefData{Datum01}{{31.12.14}{950}{0}{}}

\Fertigkeit1{Abwehr}[13]
\Fertigkeit1{Resistenz}[13]
\Fertigkeit1{Zaubern}[13]
\DefData{Abwehr}{{\StandardAbwehrPlus{0}}% Normal
	{\StandardAbwehrPlus{3}}}% Mit Verteidigungswaffe

\Fertigkeit1{Wahrnehmung}[6]
\Fertigkeit1{Trinken}[5]
\Fertigkeit1{Fechten}[7]
\Fertigkeit1{Musizieren}[12]
\Fertigkeit1{Schreiben}[8]
\Fertigkeit1{Reiten}[13]
\Fertigkeit1{Schwimmen}[13]
\Fertigkeit1{Seilkunst}[12]
\Fertigkeit1{Balancieren}[12]
\Fertigkeit1{Glücksspiel}[12]
\Fertigkeit1{Klettern}[12]
\Fertigkeit1{Etikette}[8]
\Fertigkeit1{Verführen}[10]
\Fertigkeit1{Verstellen}[10]
\Fertigkeit1{Gassenwissen}[8]
\Fertigkeit1{Menschenkenntnis}[8]

\Fertigkeit2[Fecht]{Fechtwaffe}[9]
\Fertigkeit2[waffenlos]{Waffenloser Kampf}[8]
\Fertigkeit2[Parier]{Parierwaffen}[3]
\Fertigkeit2[Schild]{Schilde}[3]
\Fertigkeit2[Stich]{Stichwaffe}[9]
\Fertigkeit2[Spiess]{Spiesswaffe}[9]

\Fertig1{Wahrnehmung}{}
\Fertig1{Trinken}{}
\Fertig1{Fechten}{Gs}
\Fertig1{Musizieren}{Gs}
\Fertig1{Schreiben}{In}
\Fertig1{Reiten}{Gw}
\Fertig1{Schwimmen}{Gw}
\Fertig1{Seilkunst}{Gs}
\Fertig1{Balancieren}{Gw}
\Fertig1{Glücksspiel}{Gs}
\Fertig1{Klettern}{St}
\Fertig1{Etikette}{In}
\Fertig1{Verführen}{pA}
\Fertig1{Verstellen}{pA}
\Fertig1{Gassenwissen}{In}
\Fertig1{Menschenkenntnis}{In}

\Waffe2[Florett (RK max. 2)]{Fecht}[2]{AB}[2W6\PlusSB{-4}][Fechten 2W6-4]
\Waffe2[Stoßspeer]{Spiess}{AB}[1W6\PlusSB{}][Sturm 2W6\PlusSB{}]
\Waffe2[Ochsenzunge]{Stich}{AB}[2W6\PlusSB{-4}]
\Waffe2[(Parier)Dolch]{Stich}{AB}[1W6\PlusSB{-1}]
\Waffe2[Parierdolch]{Parier}{}[-1AP]
\Waffe2[großer Schild]{Schild}{}
\Waffe2[Faustkampf (-/-4)]{waffenlos}{AB}[1W6\PlusSB{-3}]
\Waffe2[Ringen (-4/-8)]{waffenlos}{AB}[1W6\PlusSB{-3}]

\Output{Figurenblatt}

\DefData{Beherrschen}{}
\DefData{Erkennen}{}
\DefData{Formen}{}
\DefData{Zerstoeren}{}
\DefData{Dweomer}{}
\DefData{Bewegen}{}
\DefData{Erschaffen}{}
\DefData{Veraendern}{}
\DefData{Wundertaten}{}
\DefData{Zauberlieder}{}

\Output{Abenteuerblatt}

\Magisch{2 Amphoren Zauberöl}{}

%\Zauber1{Handauflegen}{1}{Wund./""Erschaffen}{10\,sec}{Berührung}%
%  {Körper}{1 Wesen}{0}{S.~143; 1W6 AP (1 mal pro Tag)}{Geste}

%\Output{Zauberblatt}
%\Output{ZauberblattMagier}

% Für einen Kämpfer geben wir statt des Zauberblatts ein  Beischreibungsblatt
% aus, das zusätzlich die magischen Gegenstände enthält:

\Output{BeschreibungsblattKopf}
%\Output{FigurenblattKopf}
%\Output{AbenteuerblattKopf}
%\Output{ZauberblattKopf}

\begin{minipage}{\Halbseitenbreite}%
Beschreibung:
\end{minipage}
\begin{minipage}{\Halbseitenbreite}%
\Output{magischeGegenstaende}
\end{minipage}%
\clearpage



% Am Ende muss immer \end{document} stehen. Alles danach wird ignoriert.
%\end{document}
% --- Ende ---

}% Schließen der zweiten Klammer zum "Vergessen" aller Figurendaten.

\DefData{Spieler}{}% Wir setzen auch den global definierten Spieler zurück

{% Hier könnten wir die nächste Figur definieren und ausgeben.
% Da wir hier nichts definieren, werden leere Bögen ausgegeben.

% Da die Leiteigenschaften nicht definiert sind, dürfen wir sie nicht benutzen
% (sonst erhalten wir Fehlermeldungen bei deren Referenzierung):

\UngelerntOhne

% Leere Fertigkeits-/Waffen-/Zauber-/Gegenstandslisten werden unterdrückt.
% Wir fügen daher leere Werte zu den Tabellen hinzu,
% um eine leere Tabellen zu erhalten:

\Fertigkeit1{Abwehr}
\Fertigkeit1{Resistenz}
\Fertigkeit1{Zaubern}
\Fertigkeit1[]{}
\Fertigkeit1[]{}
\Fertigkeit1[]{}
\Fertigkeit1[]{}
\Fertigkeit1[]{}
\Fertigkeit1[]{}
\Fertigkeit1[]{}
\Fertigkeit1[]{}
\Fertigkeit1[]{}
\Fertigkeit1[]{}
\Fertigkeit1[]{}
\Fertigkeit1[]{}
\Fertigkeit1[]{}
\Fertigkeit1[]{}
\Fertigkeit1[]{}
\Fertigkeit1[]{}
\Fertigkeit1[]{}
\Fertigkeit1[]{}
\Fertigkeit1[]{}
\Fertigkeit1[]{}
\Fertigkeit1[]{}
\Fertigkeit1[]{}
\Fertigkeit1[]{}
\Fertigkeit1[]{}
\Fertigkeit1[]{}
\Fertigkeit1[]{}
\Fertigkeit1[]{}
\Fertigkeit1[]{}
\Fertigkeit1[]{}
\Fertigkeit1[]{}
\Fertigkeit1[]{}
\Fertigkeit1[]{}
\Fertigkeit1[]{}
\Fertigkeit1[]{}
\Fertigkeit1[]{}
\Fertigkeit1[]{}
\Fertigkeit2[]{}
\Fertigkeit2[]{}
\Fertigkeit2[]{}
\Fertigkeit2[]{}
\Fertigkeit2[]{}
\Fertigkeit2[]{}
\Fertigkeit2[]{}
\Fertigkeit2[]{}
\Fertigkeit2[]{}
\Fertigkeit2[]{}
\Fertigkeit2[]{}
\Fertigkeit2[]{}
\Fertigkeit2[]{}
\Fertigkeit2[]{}
\Fertigkeit2[]{}
\Fertigkeit2[]{}
\Fertigkeit2[]{}
\Fertigkeit2[]{}
\Fertigkeit2[]{}
\Fertigkeit2[]{}
\Fertigkeit2[]{}
\Fertigkeit2[]{}
\Fertigkeit2[]{}
\Fertigkeit2[]{}
\Fertigkeit2[]{}
\Fertigkeit2[]{}
\Fertigkeit2[]{}
\Fertigkeit2[]{}
\Fertigkeit2[]{}
\Fertigkeit2[]{}
\Fertigkeit2[]{}
\Fertigkeit2[]{}
\Fertigkeit2[]{}
\Fertigkeit2[]{}
\Fertigkeit2[]{}

\Fertig1{}{}
\Fertig1{}{}
\Fertig1{}{}
\Fertig1{}{}
\Fertig1{}{}
\Fertig1{}{}
\Fertig1{}{}
\Fertig1{}{}
\Fertig1{}{}
\Fertig1{}{}
\Fertig1{}{}
\Fertig1{}{}
\Fertig1{}{}
\Fertig1{}{}
\Fertig1{}{}
\Fertig1{}{}
\Fertig1{}{}
\Fertig1{}{}
\Fertig1{}{}
\Fertig1{}{}
\Fertig1{}{}
\Fertig1{}{}
\Fertig1{}{}
\Fertig2{}{}
\Fertig2{}{}
\Fertig2{}{}
\Fertig2{}{}
\Fertig2{}{}
\Fertig2{}{}
\Fertig2{}{}
\Fertig2{}{}
\Fertig2{}{}
\Fertig2{}{}
\Fertig2{}{}

\Waffe1{}{}
\Waffe1{}{}
\Waffe1{}{}
\Waffe1{}{}
\Waffe1{}{}
\Waffe1{}{}
\Waffe1{}{}
\Waffe1{}{}
\Waffe1{}{}
\Waffe1{}{}
\Waffe1{}{}
\Waffe1{}{}
\Waffe1{}{}
\Waffe2{}{}
\Waffe2{}{}
\Waffe2{}{}
\Waffe2{}{}
\Waffe2{}{}
\Waffe2{}{}
\Waffe2{}{}
\Waffe2{}{}
\Waffe2{}{}
\Waffe2{}{}
\Waffe2{}{}
\Waffe2{}{}
\Waffe2{Raufen}{}

\Zauber1{}{}{}{}{}{}{}{}{}{}
\Zauber1{}{}{}{}{}{}{}{}{}{}
\Zauber1{}{}{}{}{}{}{}{}{}{}
\Zauber1{}{}{}{}{}{}{}{}{}{}
\Zauber1{}{}{}{}{}{}{}{}{}{}
\Zauber1{}{}{}{}{}{}{}{}{}{}
\Zauber1{}{}{}{}{}{}{}{}{}{}
\Zauber1{}{}{}{}{}{}{}{}{}{}
\Zauber1{}{}{}{}{}{}{}{}{}{}
\Zauber1{}{}{}{}{}{}{}{}{}{}
\Zauber1{}{}{}{}{}{}{}{}{}{}
\Zauber1{}{}{}{}{}{}{}{}{}{}
\Zauber1{}{}{}{}{}{}{}{}{}{}
\Zauber1{}{}{}{}{}{}{}{}{}{}
\Zauber2{}{}{}{}{}{}{}{}{}{}
\Zauber2{}{}{}{}{}{}{}{}{}{}
\Zauber2{}{}{}{}{}{}{}{}{}{}
\Zauber2{}{}{}{}{}{}{}{}{}{}
\Zauber2{}{}{}{}{}{}{}{}{}{}
\Zauber2{}{}{}{}{}{}{}{}{}{}
\Zauber2{}{}{}{}{}{}{}{}{}{}
\Zauber2{}{}{}{}{}{}{}{}{}{}

\Magisch{}{}
\Magisch{}{}
\Magisch{}{}
\Magisch{}{}
\Magisch{}{}
\Magisch{}{}
\Magisch{}{}
\Magisch{}{}
\Magisch{}{}
\Magisch{}{}
\Magisch{}{}
\Magisch{}{}
\Magisch{}{}
\Magisch{}{}

\Output{Figurenblatt}
\Output{Abenteuerblatt}
\Output{ZauberblattMagier}
\Output{Zauberblatt}
\Output{BeschreibungsblattKopf}\clearpage

}% Diese Klammer schließt die "leere" Figur.

% Am Ende muss immer \end{document} stehen. Alles danach wird ignoriert.
\end{document}
